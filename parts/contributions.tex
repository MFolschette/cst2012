\chapter{Contributions}

\todo{Nouvelles sémantiques : arcs neutralisants, actions conjointes\\
Traduction PH vers Thomas (article)}

\section{Nouvelles sémantiques}
Un premier volet de mon début de thèse a consisté en la recherche de nouvelles sémantiques afin d'augmenter l'expressivité du Process Hitting, afin notamment d'étudier des alternatives à l'utilisation de la sémantique de priorités fixes.

Une première proposition a consisté en l'introduction d'arcs neutralisants jouant le rôle de « priorités atomiques ».
Un arc neutralisant est un arc orienté qui relie deux actions dans un Process Hitting.
Si la première action est activée (\ie ses processus frappeur et cible sont présents), la présence de l'arc neutralisant interdit de jouer la seconde action, même si celle-ci est aussi activée.
Si les arcs neutralisants permettent d'ajouter du sens à la neutralisation d'une action par une autre, il a cependant été montré que cette sémantique est équivalente à celle des Process Hitting à priorités fixes.
De plus, leur utilisation ne permet pas de retrouver directement l'ensemble des points fixes d'un modèle ni d'utiliser les méthodes d'analyse statique.

\todo{cf. /media/DATA/Thèse/PH/Traduction\_arcs\_neutralisants-priorités/main-v1.pdf}

%L'une des utilisations les plus intéressantes de la sémantique de Process Hitting à priorités fixes est la modélisation de coopérations exactes grâce à la mise à jour prioritaire des sortes coopératives.
Étant donnée la structure restreinte des actions (qui ne concernent que deux processus au plus), une coopération entre composants peut être représentée dans un Process Hitting par une sorte particulière appelée sorte coopérative, qui permet uniquement de représenter la configuration d'un ensemble de composants.
Une sorte coopérative est cependant soumise à la même sémantique qu'une sorte standard, ce qui a pour conséquence d'introduire un décalage temporel entre l'état des sortes représentées et l'état de la sorte coopérative, menant à une sur-approximation du comportement au niveau de la coopération.
Pour pallier cela, une seconde proposition de sémantique a consisté en la généralisation de la notion d'actions à celle d'actions conjointes, permettant l'expression d'un ensemble de frappeurs plutôt qu'un seul.
Les actions conjointes permettent de donner du sens aux coopérations mais elles sont équivalentes à l'utilisation d'une sorte coopérative utilisée dans la sémantique du Process Hitting à priorités fixes, et ne permettent pas d'utiliser les méthodes de recherche de points fixes et d'analyse statique.

\todo{cf. /media/DATA/Thèse/PH/Traduction\_RRB-actions\_conjointes/main.pdf}

\section{Traduction d'un Process Hitting en modèle de Thomas}
Dans un second volet du début de ma thèse, j'ai été amené à effectuer un stage au National Institute of Informatics à Tokyo, dans l'équipe du professeur Katsumi Inoue.
Le sujet initial du stage portait sur l'utilisation de la programmation logique par contraintes afin de déterminer des paramètres discrets du modèle de Thomas depuis un Process Hitting, mais il a été généralisé en une méthode permettant la concrétisation d'un Process Hitting en un modèle de Thomas complet.

Cette méthode en deux temps repose tout d'abord sur une recherche exhaustive des influences sur un composant de ses régulateurs, en prenant en compte les coopérations possibles.
Elle mène à la détermination du Graphe des Interactions représenté par le Process Hitting et à l'inférence des paramètres discrets nécessaires.
Cette étape peut ne pas aboutir dans le cas où la dynamique du modèle représenté en Process Hitting ne peut pas être représentée par un modèle de Thomas ;
ces cas non-conclusifs permettent cependant d'établir un Graphe des Interactions partiel (comportant des influences non étiquetées), ou à une paramétrisation partielle (dont certains paramètres ne peuvent être inférés).
La seconde étape consiste en une énumération de l'ensemble des paramétrisations compatibles avec les paramètres inférés et la dynamique du Process Hitting.
Le comportement de tous les modèles de Thomas obtenus grâce à cette méthode est nécessairement inclus dans le comportement du Process Hitting initial.
Ces différentes étapes ont été programmés à l'aide de programmation logique (Answer Set Programming) et intégrées à la bibliothèque Pint\footnote{Disponible à \url{http://process.hitting.free.fr}} développée par Loïc Paulevé.

\todo{Ce travail a donné lieu à la publication d'un papier qui est reproduit en annexe.\\ref + cf. /home/max/pint/cmsb/main.pdf}

\todo{Il a aussi fait l'objet d'une seconde soumission à un workshop + ref}


