\chapter{Autres activités}

\section{Enseignement}

Activité complémentaire d'enseignement (monitorat) à l'École Centrale de Nantes ; les heures de TP sont décomptées comme des heures équivalent TD.

\bigskip

\noindent
\textbf{Méthodes logicielles (MELOG) :} 2\textsuperscript{e} année (semestre 7)\\
Programmation orientée objet, structures de données et langage Java\\
\textbf{Responsable :} Guillaume MOREAU
\begin{itemize}
  \item 2 groupes de TP, soit 28 heures
\end{itemize}

\bigskip\noindent
\textbf{Algorithmique et programmation (ALGPR) :} 1\textsuperscript{ère} année (semestre 6)\\
Introduction à l'algorithmique et applications au langage C\\
\textbf{Responsable :} Vincent TOURRE
\begin{itemize}
  \item 2 modules de TD, soit 4 heures
  \item 2 modules de TP, soit 4 heures
\end{itemize}

\bigskip\noindent
\textbf{Systèmes d'information et bases de données (dSIBAD) :} 1\textsuperscript{ère} année (semestre 6)\\
Conception et utilisation de bases de données et langage SQL\\
\textbf{Responsable :} Morgan MAGNIN
\begin{itemize}
  \item 2 modules de TD, soit 4 heures
  \item 2 groupes de TP, soit 16 heures
  \item encadrement de 6 groupes de 4 à 5 élèves pour des projets en autonomie, soit 4 heures
\end{itemize}

\bigskip\noindent
\textbf{Projet de Recherche et Développement (PRORD) :} 3\textsuperscript{e} année, option R\&D (semestre~8)\\
Initiation à la recherche, projet de recherche\\
\textbf{Responsable :} Ina TARALOVA
\begin{itemize}
  \item encadrement de 2 projets pour un total de 3 étudiants en co-encadrement, soit 4,16 heures
\end{itemize}

\bigskip\noindent
\textbf{Projet d'application (PAPPL) :} 3\textsuperscript{e} année, option informatique (semestre 8)\\
Projet d'application informatique\\
\textbf{Responsable :} Didier LIME
\begin{itemize}
  \item encadrement de 2 projets pour un total de 3 étudiants en co-encadrement, soit 4,6 heures
\end{itemize}



\section{Actions de diffusion de la culture scientifique}

\paragraph{Sensibilisation au monde de la recherche}
\begin{itemize}
\item[]
Dans le cadre de l'électif « Projet d'études et de recherche »,
participation à une séance d'une heure d'échanges auprès d'étudiants de première année sur les enjeux et débouchés de la recherche.
\end{itemize}

\paragraph{Participation à un concours de rédaction pour jeunes chercheurs}
\begin{itemize}
\item[] Rédaction d'un article de blog\footnote{Disponible à \scriptsize\url{http://en.bioinformatyk.eu/contest-articles/a-presentation-of-thomas-modelling.html}} pour \texttt{en.bioinformatyk.eu} présentant le modèle de Thomas de façon accessible dans le cadre d'un concours destiné à de jeunes chercheurs en bioinformatique.
\end{itemize}



\section{Responsabilité administrative}
Membre actif du bureau et responsable informatique au sein de l'\emph{Association des Étudiants en Doctorat de l'École Centrale de Nantes} (AED).
